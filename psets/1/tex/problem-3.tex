\begin{statement}{3}
  Sea $(X, Y)$ un vector aleatorio discreto que refleja la siguiente informaci\'on:
  $X = 1$ si se da un shock positivo de oferta en la econom\'ia,
  $X = 0$ si la econom\'ia no sufre ning\'un shock,
  $X = -1$ si el shock es negativo,
  $Y = 1$ si el nivel del empleo aumenta,
  $Y = 0$ si se mantiene y
  $Y = -1$ si disminuye.
  La distribuci\'on de probabilidades conjunta est\'a dada por la siguiente tabla:
  \begin{center}
    \begin{tabular}{ | c | c | c | c | c | c | }
      \hline
      X \textbackslash \, Y & -1 & 0 & 1 \\ \hline
      -1 & 5/24 & 3/24 & 0 \\ \hline
      0 & 2/24 & 6/24 & 2/24 \\ \hline
      1 & 1/24 & 2/24 & 3/24 \\
      \hline
    \end{tabular}
  \end{center}
\end{statement}

\begin{statement}{a}
  Si tener informaci\'on acerca del shock econ\'omico.
  ¿Qu\'e es m\'as probable, que aumente o que disminuya el nivel de empleo?
\end{statement}

\begin{solution}
  Sumando los valores en cada columna de la tabla calculamos
  \[
    \BP[Y = y, X \in \{-1, 0, 1\}] = \begin{cases}
      8/24 & y = -1, \\
      11/24 & y = 0, \\
      5/24 & y = 1.
    \end{cases}
  \]
  Observamos que es m\'as probable que el nivel de desempleo se mantenga.
  Sin embargo, si solo nos fijamos en el aumento o la disminuci\'on, es m\'as probable que disminuya el empleo.
\end{solution}

\begin{statement}{b}
  Si el schock ha sido negativo, ¿cambiar\'ia la respuesta en (a)?
  ¿Cu\'a ser\'ia la probabilidad de que el empleo caiga, que se mantenga o que suba?
\end{statement}

\begin{solution}
  Primero determinemos cu\'al es la probabilidad de que el shock sea negativo:
  \[
    \BP[X = -1, Y \in {-1, 0, 1}\} = \frac{5}{24} + \frac{3}{24} + 0 = \frac{1}{3}.
  \]
  Ahora calculemos la funci\'on de densidad del nivel de desempleo dado que el shock es negativo
  \[
    \BP[Y = y | X = -1] = \frac{\BP[Y = y \wedge X = -1]}{\BP[X = -1]} = \begin{cases}
      5/8 & y = -1, \\
      3/8 & y = 0, \\
      0 & y = 1.
    \end{cases}
  \]
  Observamos que la respuesta en (a) cambi\'o puesto que es m\'as probable
  que el nivel de empleo disminuya dado que el shock fue negativo considerando las tres posibilidades.
  Si solo consideramos el aumento y la disminuci\'on, la tendencia se mantiene, es decir, sigue siendo m\'as probable
  que el empleo disminuya a que aumente.
\end{solution}

\begin{statement}{c}
  Sin tener informaci\'on acerca del shock de la econom\'ia,
  ¿cu\'al es el valor esperado de la variable que describe la evoluci\'on del empleo?
\end{statement}

\begin{solution}
  De (a) obtenemos
  \[
    \BE[Y] = \sum_{y = -1}^1 y \, \BP[Y = y, X \in \{-1, 0, 1\}] = -\frac{1}{8}.
  \]
\end{solution}

\begin{statement}{d}
  ¿Cu\'al ser\'ia el valor esperado de la variable que representa el movimiento del empleo si el shock ha sido negativo?
\end{statement}

\begin{solution}
  De (b) obtenemos
  \[
    \BE[Y] = \sum_{y = -1}^1 y \, \BP[Y = y, X = -1] = -\frac{5}{8}.
  \]
\end{solution}